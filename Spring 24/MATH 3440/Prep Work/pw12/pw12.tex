\documentclass{article}
\usepackage{amsmath, amssymb} % Required for math symbols
\usepackage[paper=letterpaper,
           hmargin={1in,1in},
           vmargin={1in,1in},
           ]{geometry}   % Allows you to change the margin sizes
\usepackage{enumitem}  % Required to re-label lists

\title{Prep Work 12}
\author{Xander} 
\date{Mar 26}

\begin{document}

\maketitle
\noindent\textbf{Presentation: } 
%%%%%%%%%%%%%%%%% Don't delete anything above this line!

\section*{Exercise 5 3.2}  

Prove that for all integers \(n\), it is the case that \(n\) is  even if and only if \(3n\) is even.  That is, prove both implications: if \(n\) is even, then \(3n\) is even, and if \(3n\) is even, then \(n\) is even.

\vspace{0.5cm}
\noindent\textbf{Solution Draft:} 
\vspace{0.2cm}

\textbf{If \(n\) is even, then \(3n\) is even.}


Assuming \(n\) is even. Then, there exists an integer \(k\) such that \(n = 2k\). Consider the product \(3n\):
\begin{align*}
    3n &= 3(2k) \\
    &= 6k \\
    &= 2(3k)
    \end{align*}
    
Since \(3k\) is an integer, \(6k\) is even, proving the first part.
\vspace{0.2cm}

\textbf{If \(3n\) is even, then \(n\) is even.}


Assuming \(3n\) is even. Then, there exists an integer \(m\) such that \(3n = 2m\). To prove \(n\) is even by contradiction, we assume \(n\) is odd, so \(n = 2k + 1\) for some \(k\). Then, 
\begin{align*}
    3n &= 3(2k + 1) \\
    &= 6k + 3 \\
    &= 2(3k) + 3
    \end{align*}    
This is not divisible by \(2\), a contradiction. Therefore, \(n\) must be even.


%%%%%%%%%%%%%%%%%%%%%%%%%%%%%%%%%%%%
\section*{Exercise 6 3.2}  

Prove that \(\sqrt 3\) is irrational.


\vspace{0.5cm}
\noindent\textbf{Solution Draft:} 
\vspace{0.2cm}

To begin, we will assume that \(\sqrt{3}\) is rational, meaning it can be expressed as a fraction \(\frac{a}{b}\), where \(a\) and \(b\) are integers with no common factor other than 1, and \(b \neq 0\).

Given this assumption, we have:
\[\sqrt{3} = \frac{a}{b}\]

Squaring both sides yields:
\[3 = \frac{a^2}{b^2}\]

Rearranging gives:
\[a^2 = 3b^2\]

This implies that \(a^2\) is a multiple of 3. For \(a^2\) to be a multiple of 3, \(a\) itself must also be a multiple of 3 as the square of a non-multiple of 3 cannot be a multiple of 3. Let us denote \(a\) as \(3k\), where \(k\) is an integer.

Substituting \(3k\) for \(a\) in the equation \(a^2 = 3b^2\) we get:
\begin{align*}
    (3k)^2 &= 3b^2 \\
    9k^2 &= 3b^2 \\
    b^2 &= 3k^2
    \end{align*}
    

This shows that \(b^2\) is also a multiple of 3, and hence \(b\) must also be a multiple of 3.

However, since both \(a\) and \(b\) are multiples of 3, they share a common factor greater than 1. This contradicts our initial assertion that \(a\) and \(b\) have no common factor other than 1. Therefore, our assumption that \(\sqrt{3}\) is rational must be false.


%%%%%%%%%%%%%%%%%%%%%%%%%%%%%%%%%%%%
\section*{Exercise 14 3.2}  

Prove that there are no integer solutions to the equation \(x^2 = 4y + 3\).


\vspace{0.5cm}
\noindent\textbf{Solution Draft:} 
\vspace{0.2cm}

To start, we assume integers $x$ and $y$ exist and complete the equation:

\[x^2 = 4y + 3\]

On the right side, $4y + 3$, indicates that for any integer $y$, $4y$ is divisible by 4, and adding 3 to it makes $4y + 3$ equal to one more than a multiple of 4. Thus, dividing $4y + 3$ by 4 leaves a remainder of 3.

On the left side, the remainder when an integer squared is divided by 4 can only be 0 or 1. This is because:
- If $x$ is even, then $x^2 = (2k)^2 = 4k^2$; divisible by 4.
- If $x$ is odd (say $x = 2k + 1$), this leaves a remainder of 1 when divided by 4.
\begin{align*}
    x^2 &= (2k + 1)^2 \\
    &= 4k^2 + 4k + 1 \\
    &= 4(k^2 + k) + 1
    \end{align*}


Therefore, a squared integer can never have a remainder of 3 after divided by 4. This contradicts the original statement. So, there are no integer solutions to the equation $x^2 = 4y + 3$.

%%%%%%%%%%%%%%%%%%%%%%%%%%%%%%%%%%%%
\end{document}
