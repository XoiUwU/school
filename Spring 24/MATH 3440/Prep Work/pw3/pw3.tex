% The percent symbol creates a comment, which will be ignored by LaTeX when it creates your document.  They're useful for leaving yourself notes and organizing your work.
\documentclass{article}
\usepackage{amsmath, amssymb} % Required for math symbols
\usepackage[paper=letterpaper,
           hmargin={1in,1in},
           vmargin={1in,1in},
           ]{geometry}   % Allows you to change the margin sizes
\usepackage{enumitem}  % Required to re-label lists

\title{Prep Work 3}
\author{Xander} 
\date{Jan 30}

\begin{document}

\maketitle


%%%%%%%%%%%%%%%%% Don't delete anything above this line!

\section*{Exercise 0.3 10}  

\noindent\textbf{Presentation:} Yes

\vspace{0.5cm} % Creates vertical space

\noindent\textbf{Question} 
Let $A = \{x \in 3 \le x \le 13\}$, $B = \{x \in x \mbox{ is even} \}$ is even, and $C = \{x \in x \} $ is odd.
\begin{enumerate}
    \item Find $A \cap B$.
    
    \item Find $A \cup B$.

    \item Find $B \cap C$.

    \item Find $B \cup C$.
\end{enumerate}
\noindent\textbf{Solution Draft:} 
\begin{enumerate}
    \item $A \cap B=\{4,6,8,10,12\}$
    
    Since $A$ is the set of integers from 3 to 13, and $B$ is the set of all even integers, $A \cap B$ will be the even numbers from 4 to 12.
    \item $A \cup B=A$

    Since $A$ already includes both even and odd numbers from 3 to 13, and $B$ includes all even numbers, $A \cup B$ will essentially be the same as the set of all integers because every integer is either within the range of $A$ or is even.

    \item $B \cap C = \emptyset$

    This set contains elements that are both even and odd, which is an impossibility because no integer can be both even and odd

    \item $B \cup C= \{ x \in x \}$
    
    Since every integer is either even or odd, the set contains all integers.

\end{enumerate}


%%%%%%%%%%%%%%%%%%%%%%%%%%%%%%%%%%%%
\section*{Exercise 0.3 24}  

\noindent\textbf{Presentation:} No

\vspace{0.5cm} % Creates vertical space

\noindent\textbf{Question}
Let \( X = \{n \in \mathbb{N} : 10 \leq n < 20\} \). Find examples of sets with the properties below and very briefly explain why your examples work.

\begin{enumerate}
    \item[a.] A set \( A \subseteq \mathbb{N} \) with \( |A| = 10 \) such that \( X \setminus A = \{10, 12, 14\} \).
    \item[b.] A set \( B \in \mathcal{P}(X) \) with \( |B| = 5 \).
    \item[c.] A set \( C \subseteq \mathcal{P}(X) \) with \( |C| = 5 \).
    \item[d.] A set \( D \subseteq X \times X \) with \( |D| = 5 \)
    \item[e.] A set \( E \subseteq X \) such that \( |E| \in E \).
\end{enumerate}

\noindent\textbf{Solution Draft:} 

\begin{enumerate}
    \item[a.] \( A = \{11, 13, 15, 16, 17, 18, 19\} \). 
    
    This works because \( A \) contains 7 elements from \( X \), and when you remove \( A \) from \( X \), you're left with \( \{10, 12, 14\} \), as required.

    \item[b.] \( B = \{10, 11, 12, 13, 14\} \). 
    
    \( B \) is a subset of \( X \) and has 5 elements, satisfying the condition \( |B| = 5 \).

    \item[c.] \( C = \{\{10\}, \{11\}, \{12\}, \{13\}, \{14\}\} \). 
    
    Each subset in \( C \) is an element of \( \mathcal{P}(X) \), and there are 5 subsets, so \( |C| = 5 \).

    \item[d.] \( D = \{(10,11), (11,12), (12,13), (13,14), (14,15)\} \). 
    
    This subset of \( X \times X \) has 5 ordered pairs, so \( |D| = 5 \).

    
    \item[e.] Set \( E = \{11\} \). Here, \( |E| = 1 \) and \( 1 \notin X \), but since \( 11 \in X \) and \( 11 \in E \), the condition \( |E| \in E \) is met.
\end{enumerate}

%%%%%%%%%%%%%%%%%%%%%%%%%%%%%%%%%%%%
\section*{Exercise 0.3 25}  

\noindent\textbf{Presentation:} No

\vspace{0.5cm} % Creates vertical space

\noindent\textbf{Question}
25. Let \( A, B \) and \( C \) be sets.

\begin{enumerate}
    \item[a.] Suppose that \( A \subseteq B \) and \( B \subseteq C \). Does this mean that \( A \subseteq C \)? Prove your answer. Hint: to prove that \( A \subseteq C \) you must prove the implication, ``for all \( x \), if \( x \in A \) then \( x \in C \)''.

    \item[b.] Suppose that \( A \in B \) and \( B \in C \). Does this mean that \( A \in C \)? Give an example to prove that this does NOT always happen (and explain why your example works). You should be able to give an example where \( |A| = |B| = |C| = 2 \).
\end{enumerate}

\noindent\textbf{Solution Draft:} 
\begin{enumerate}
\item[a.] Yes, if \( A \subseteq B \) and \( B \subseteq C \), then \( A \subseteq C \). 

We assume that \( A \subseteq B \) and \( B \subseteq C \), and we need to show that for any \( x \), if \( x \in A \) then \( x \in C \). Since \( A \subseteq B \), for any \( x \in A \), it follows that \( x \in B \). And since \( B \subseteq C \), it must be that \( x \in C \). Therefore, \( A \subseteq C \).

\item[b.] No, \( A \in B \) and \( B \in C \) does not necessarily mean that \( A \in C \). 

Let \( A = \{1, 2\} \), \( B = \{\{1, 2\}, 3\} \), and \( C = \{\{\{1, 2\}, 3\}, 4\} \). Here, \( A \in B \) because \( A \) is one of the elements of \( B \), and \( B \in C \) because \( B \) is one of the elements of \( C \). However, \( A \) is not an element of \( C \); rather, \( A \) is a subset of one of the elements of \( C \), which is \( B \). Thus, \( A \notin C \).
\end{enumerate}


\section*{Exercise 0.3 29}  

\noindent\textbf{Presentation:} No

\vspace{0.5cm} % Creates vertical space

\noindent\textbf{Question}
Explain why there is no set \( A \) which satisfies \( A = \{2, |A|\} \).

\vspace{0.5cm} % Creates vertical space

\noindent\textbf{Solution Draft:}

Assuming such a set \( A \) exists, it must satisfy the condition that \( A = \{2, |A|\} \). This means that the set \( A \) has two elements, namely 2 and the cardinality of \( A \) itself. 

If \( A \) has two elements, then \( |A| = 2 \). But then \( A \) would have to be \( \{2, 2\} \), which simplifies to \( \{2\} \) since sets do not account for duplicate elements, and hence \( |A| \) would be 1, not 2. Thus, \( A \) cannot simultaneously have two distinct elements and have its cardinality as one of its elements. Therefore, there is no such set \( A \) that satisfies the given condition.


\section*{Investiage!}


\noindent\textbf{Presentation:} No

\vspace{0.5cm} % Creates vertical space

\noindent\textbf{Question 4}
Let \( A = \{1,2,\ldots,10\} \). Define \( B_2 = \{B \subseteq A : |B| = 2\} \). Find \( |B_2| \).

\vspace{0.5cm} % Creates vertical space

\noindent\textbf{Solution Draft:}

The set \( A = \{1,2,\ldots,10\} \) contains 10 elements. The number of 2-element subsets of \( A \) is given by the binomial coefficient \(\binom{10}{2}\), which represents the number of ways to choose 2 elements from 10 without regard to order. This can be calculated as:
\[
\binom{10}{2} = \frac{10!}{2!(10-2)!} = \frac{10 \times 9}{2 \times 1} = 45.
\]
Therefore, \( |B_2| = 45 \).


\noindent\textbf{Presentation:} No

\vspace{0.5cm} % Creates vertical space

\noindent\textbf{Question 5}
For any sets \( A \) and \( B \), define \( AB = \{ab : a \in A \land b \in B\} \). If \( A = \{1,2\} \) and \( B = \{2,3,4\} \), what is \( |AB| \)? What is \( |A \times B| \)?

\vspace{0.5cm} % Creates vertical space

\noindent\textbf{Solution Draft:}

Given \( A = \{1,2\} \) and \( B = \{2,3,4\} \), the set \( AB \) is defined as \( AB = \{ab : a \in A \land b \in B\} \). Thus, we have:
\[
AB = \{1 \times 2, 1 \times 3, 1 \times 4, 2 \times 2, 2 \times 3, 2 \times 4\} = \{2, 3, 4, 4, 6, 8\} = \{2, 3, 4, 6, 8\}.
\]
Therefore, \( |AB| = 5 \).

The Cartesian product \( A \times B \) is the set of all ordered pairs where the first element comes from \( A \) and the second element comes from \( B \). Thus:
\[
A \times B = \{(1,2), (1,3), (1,4), (2,2), (2,3), (2,4)\}.
\]
Therefore, \( |A \times B| = 6 \).


\end{document}
