% The percent symbol creates a comment, which will be ignored by LaTeX when it creates your document.  They're useful for leaving yourself notes and organizing your work.
\documentclass{article}
\usepackage{amsmath, amssymb} % Required for math symbols
\usepackage[paper=letterpaper,
           hmargin={1in,1in},
           vmargin={1in,1in},
           ]{geometry}   % Allows you to change the margin sizes
\usepackage{enumitem}  % Required to re-label lists

\title{Prep Work Template}
\author{Add your name here by editing the preamble} 
\date{Add the date here by editing the preamble}

\begin{document}

\maketitle


%%%%%%%%%%%%%%%%% Don't delete anything above this line!

\section*{Chapter 0.4 Exercise 18}  

\noindent\textbf{Presentation:} Include here whether you'd be willing to present this one. 

\vspace{0.5cm} % Creates vertical space

\noindent\textbf{Question}
What can you deduce about the sets \(X\) and \(Y\) if you know,

\begin{enumerate}
    \item[a.] there is an injective function \(f : X \rightarrow Y\)? Explain.
    \item[b.] there is a surjective function \(f : X \rightarrow Y\)? Explain.
    \item[c.] there is a bijective function \(f : X \rightarrow Y\)? Explain.
\end{enumerate} 

\noindent\textbf{Solution Draft:} 

\begin{enumerate}
    \item[a.] In the injective function \(f : X \rightarrow Y\), no two elements in $X$ map to the same element in $Y$.
    \item[b.] In the surjective function\(f : X \rightarrow Y\), every element in $Y$ is assigned to at least one element of $X$
    \item[c.] In the bijective function\(f : X \rightarrow Y\), every element in $X$ is assigned to exactly one element in $Y$.
\end{enumerate} 

%%%%%%%%%%%%%%%%%%%%%%%%%%%%%%%%%%%%
\section*{Chapter 0.4 Exercise 19}  

\noindent\textbf{Presentation:} Include here whether you'd be willing to present this one. 

\vspace{0.5cm} % Creates vertical space

\noindent\textbf{Question}
Suppose \(f : X \rightarrow Y\) is a function. Which of the following are possible? Explain.

\begin{itemize}
    \item[a.] \(f\) is injective but not surjective.
    \item[b.] \(f\) is surjective but not injective.
    \item[c.] \(|X| = |Y|\) and \(f\) is injective but not surjective.
    \item[d.] \(|X| = |Y|\) and \(f\) is surjective but not injective.
    \item[e.] \(|X| = |Y|\), \(X\) and \(Y\) are finite, and \(f\) is injective but not surjective.
    \item[f.] \(|X| = |Y|\), \(X\) and \(Y\) are finite, and \(f\) is surjective but not injective.
\end{itemize}

\noindent\textbf{Solution Draft:} 

\begin{itemize}
    \item[a.] When \(f\) is injective but not surjective, $f$ is possible when $|X| \le |Y|$.
    \item[b.] When \(f\) is surjective but not injective, $f$ is possible when $|X| \ge |Y|$.
    \item[c.] When \(|X| = |Y|\) and \(f\) is injective but not surjective, $f$ is not possible as \(|X| = |Y|\) means that it is bijunctive.
    \item[d.] When \(|X| = |Y|\) and \(f\) is surjective but not injective, $f$ is not possible as \(|X| = |Y|\) means that it is bijunctive.
    \item[e.] When \(|X| = |Y|\), \(X\) and \(Y\) are finite, and \(f\) is injective but not surjective, $f$ is not possible as \(|X| = |Y|\) means that it is bijunctive.
    \item[f.] When \(|X| = |Y|\), \(X\) and \(Y\) are finite, and \(f\) is surjective but not injective, $f$ is not possible as \(|X| = |Y|\) means that it is bijunctive.
\end{itemize}

%%%%%%%%%%%%%%%%%%%%%%%%%%%%%%%%%%%%
\section*{Chapter 0.4 Exercise 23}  

\noindent\textbf{Presentation:} Include here whether you'd be willing to present this one. 

\vspace{0.5cm} % Creates vertical space

\noindent\textbf{Question}
In the game of Hearts, four players are each dealt 13 cards from a deck of 52. Is this a function? If so, what sets make up the domain and codomain, and is the function injective, surjective, bijective, or neither?


\noindent\textbf{Solution Draft:} 

The domain of this function would be all the cards in the deck

The codomain would be the set of players' hands. In hearts, there are four players so the codomain is defined as a set of four elements, where each element is a player's hand.

Let's say the deck is $|D| = 52$ and the players' hands is $|C| = 52$

Injective
Since $|D| = |C|$, the function is injective.

Surjective
Since $|D| = |C|$, the function is surjective.

Bijective
Since $|D| = |C|$ is both injective and surjective, the function is bijective.

\end{document}
