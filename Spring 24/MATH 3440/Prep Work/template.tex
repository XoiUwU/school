% The percent symbol creates a comment, which will be ignored by LaTeX when it creates your document.  They're useful for leaving yourself notes and organizing your work.
\documentclass{article}
\usepackage{amsmath, amssymb} % Required for math symbols
\usepackage[paper=letterpaper,
           hmargin={1in,1in},
           vmargin={1in,1in},
           ]{geometry}   % Allows you to change the margin sizes
\usepackage{enumitem}  % Required to re-label lists

\title{Prep Work Template}
\author{Add your name here by editing the preamble} 
\date{Add the date here by editing the preamble}

\begin{document}

\maketitle


%%%%%%%%%%%%%%%%% Don't delete anything above this line!

\section*{Exercise 0.0}  

\noindent\textbf{Presentation:} Include here whether you'd be willing to present this one. 

\vspace{0.5cm} % Creates vertical space


\noindent\textbf{Solution Draft:} 
Solutions will always contain some words first!  At some point, we might want to add some symbols and/or equations.  We can do that by using single dollar signs like this: $x^2 - 1$, or slash-parentheses like this: \(z \neq 5\).  For longer equations, we can use slash-square brackets to set the equation on it's own line, like this:
\[ \sum_{n=1}^\infty \frac{1}{n^2} = \frac{n(n+1)(2n+1)}{6}. \]
Don't forget, \LaTeX is very Google-able!  DeTexify is a good first place to try looking up the command for a particular symbol. 

If doing a multi-part problem or breaking a proof up into cases, you may want to use a numbered list.  You can create a list that's automatically numbered by using the enumerate environment:
\begin{enumerate}
    \item first thing on the list
    \item second thing on the list
    \item third thing on the list
\end{enumerate}

\noindent You can also create lists with letters:
\begin{enumerate}[label=\alph*.]
    \item first thing on the list
\end{enumerate}
\noindent or, roman numerals:
\begin{enumerate}[label=\roman*.]
    \item first thing on the list
\end{enumerate}

For your prep work submission, delete everything from the maketitle command, up to and including this line.  You can use the next lines to copy-paste a section for each exercise and fill in the relevant pieces. 

%%%%%%%%%%%%%%%%%%%%%%%%%%%%%%%%%%%%
\section*{Exercise 0.0}  

\noindent\textbf{Presentation:} Include here whether you'd be willing to present this one. 

\vspace{0.5cm} % Creates vertical space


\noindent\textbf{Solution Draft:} 

%%%%%%%%%%%%%%%%%%%%%%%%%%%%%%%%%%%%
\section*{Exercise 0.0}  

\noindent\textbf{Presentation:} Include here whether you'd be willing to present this one. 

\vspace{0.5cm} % Creates vertical space


\noindent\textbf{Solution Draft:} 

\end{document}
